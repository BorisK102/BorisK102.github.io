


\documentclass[potrait]{beamer}
%\documentclass[t,handout]{beamer}

%%%%%%%%%%%BEAMER%%%%%%%%%%%%%%%%%%%%%%%%%%%%%%%%%
%\usepackage{graphics}       % Packages to allow inclusion of graphics
\usepackage{color} 
%\usepackage{hyperref} % for themes, etc.
\usepackage{pgf}
%\usepackage{pdfpages}
\usepackage{graphicx}
\usepackage{xcolor}
%\usepackage{times}  % fonts are up to you
%\usepackage{epsfig}


%\input{epsf.sty}


\usepackage[english]{babel}

% \usepackage{pgfpages}
% \pgfpagelayout{2 on 1}[a4paper,border shrink=5mm]

%\
\mode<presentation>
%\usetheme{default}
%\setbeamercovered{transparent=22}
%\usetheme[secheader]{Berkeley}
%\usetheme[secheader]{Boadialla}
\usetheme{Warsaw}



% \setbeamercolor{title}{fg=white, bg=blue!70!black}
% %\setbeamercolor{father}{fg=white, bg=green!20!black}
% %\setbeamercolor{child}{fg=white, bg=green!20!black}
% \setbeamercolor{section in toc}{fg=blue!70!black, bg=white}
% \setbeamercolor{frametitle}{fg=blue!70!black, bg=white}




%%%%%%%%%%%%%%%%%%%%%%%%%%%%%%%%%%%%%%%%%%%%%%%%%%%%%%%%%%%%%%%%%%%%%

\usepackage{amsfonts}
\usepackage{amssymb}
\usepackage{amsmath,amscd}

\usepackage{euscript}

%\usepackage[dvips]{epsfig,graphics}
\usepackage{latexsym, amsfonts, longtable}
%\usepackage{epsfig} % for figures
%\usepackage{psfrag}  % for writing mathematics inside the figure
%\usepackage{portland}

     \newtheorem{proposition}[theorem]{Proposition}
      \newtheorem{remark}[theorem]{Remark}

\newcommand{\s}{\mathbb S}
\newcommand{\R}{\mathbb R}
\newcommand{\Hp}{\mathbb H}
\newcommand{\Rk}{\mathbb R^k}
%\newcommand{\R^m}{\mathbb R^m}
\newcommand{\Rn}{\mathbb R^n}
\newcommand{\C}{\mathbb C}
\newcommand{\Z}{\mathbb Z}
\newcommand{\N}{\mathbb N}
\newcommand{\Zk}{\mathbb Z^k}
\newcommand{\Zn}{\mathbb Z^n}
\newcommand{\Q}{\mathbb Q}
\newcommand{\T}{\mathbb T}
\newcommand{\m}{\mathbb \mu ^F}
\newcommand{\Tm}{\mathbb T^m}
\newcommand{\Tn}{\mathbb T^n}

\newcommand{\A}{\mathcal A}
\newcommand{\Gm}{GL(m,\mathbb R)}
\newcommand{\Rmx}{\mathbb R^m_x}
\newcommand{\la}{\lambda}
\newcommand{\La}{\Lambda}

\def\r{\mathcal R}
\def\rel{\mathcal R_{\varepsilon,l}}
\def\RM{\mathcal R^\mu}
\def\relm{\mathcal R_{\varepsilon,l}^\mu}



\def\pf{\hfill\hfill{$\square$}} 
\def\A{\EuScript{A}} 
\def\B{\EuScript{B}} 
\def\E{\mathcal{E}}
\def\F{\tilde F}
\def\G{\mathcal{G}}
\def\M{\mathcal{M}}
\def\W{\mathcal{W}}
\def\o{\mathcal O}
\def\c{\EuScript{C}}
\def\R{\mathbb R}
\def\rd{{\mathbb R ^d}}   
\def\C{\mathbb C}
\def\Z{\mathbb Z}
\def\N{\mathbb N}
\def\T{\mathbb T}
\def\S{\mathbb S}
\def\dist{\text{dist}}
\def\diam{\text{diam}}
\def\Id{\text{Id}}
\def\e{\epsilon}
\def\a{\alpha}
\def\b{\beta}
\def\la{\lambda}
\def\Ci{C^\infty}

\def\u{\mathbf v}



\title{Linear cocycles over hyperbolic systems:\\
 periodic data and rigidity}
\author[Boris Kalinin]
{{\Large Boris Kalinin} }
\date{}




\begin{document}

%this prints title, author etc. info from above
\begin{frame}
\titlepage
\end{frame}


%%%%%%%%%%%%%%%%%%%%%%%%%%%%%%%%%%

\begin{frame}[t]\frametitle{ Anosov diffeomorphisms }

$f$ -- a diffeomorphism of a compact
Riemannian manifold $\M$. 
\vskip.2cm
%\pause

{\color{blue}{\bf Definition:}} $\;f$ is {\bf Anosov}\, if there exist 

a continuous invariant decomposition $T\M=E^s\oplus E^u$

 and constants $K>0$, 
$\lambda>0$ such that for all $n\in \N$,
\vskip.2cm
$\hskip2cm  \| Df^n(v) \|  \leq K e^{-\lambda n} \| v \|
     \quad\text{for all }v \in E^s , $
 \vskip.1cm    
 $ \hskip1.8cm \| Df^{-n}(v) \| \leq K e^{-\lambda n} \| v \|
     \quad\text{for all }v \in E^u. $

\vskip.2cm     
 $E^s$ and $E^u$ -- \,stable and unstable sub-bundles.   
%\vskip.15cm
%$f$ is  {\bf transitive} if there exists a point in $\M$ with  dense orbit. 
\vskip.3cm

\pause    
{\color{blue}{\bf Basic examples:}} Anosov automorphisms of tori.
\vskip.05cm

$A$ -- a hyperbolic matrix in $SL(d,\Z)$ (no eigenvalue of modulus 1)
\vskip.05cm
$A:\R^d\to \R^d$ projects to an automorphism  of $\;\T^d=\R^d/\Z^d$.  
%\vskip.3cm

%%\pause 
%{\color{blue}{\bf Structural stability:}} \, If $f:\M\to \M$ is Anosov then \\
%any $C^1$ small perturbation $g$ of $f$ is also Anosov and \\
%topologically conjugate to $f$: $\;\;\;g=h^{-1}\circ f \circ h$.

\end{frame}

%%%%%%%%%%%%%%%%%%%%%%%%%%%%%%%%%%

\begin{frame}[t]\frametitle{Periodic points}


Let $f: \M \to \M\,$ be a transitive Anosov diffeomorphism. 
\vskip.2cm 

%%\pause
%{\color{blue} $\bullet$} Periodic points of $f$ are dense in $\M$.
%\vskip.15cm 

%\pause

{\color{blue} {\bf Anosov Closing Lemma.}}\, If 
$\,\dist (f^n x,x) \le \e$, then there exists

 $p=f^np\,$ such that  $\,\dist (f^i p, f^i x)\le C\e\,$
 for $i=0, \dots , n$.

\pause

\vskip.3cm
{\color{blue} {\bf Liv\v{s}ic Theorem.}}  
Let $\a:\M\to \R\,$ be a H\"older function. Then
\vskip.2cm
\hskip3cm $\a(x)=\varphi(fx)-\varphi(x) \qquad (\ast) $
\vskip.2cm
has a H\"older continuous solution $\varphi$
if and only if whenever $f^n p=p$ 
\vskip.2cm
\hskip3cm$\,\sum_{i=0}^{n-1} \;\a(f^ip)\,=\,0\,$
 \vskip.2cm
Moreover, any measurable solution of $(\ast)$ is H\"older continuous .
\vskip.3cm

\pause
{\color{blue}{\bf Corollary.}}\, 
$f$ preserves a volume 
iff $\; \det Df^n(p)= 1\,$  for $\,f^n p=p$.


\end{frame}

%%%%%%%%%%%%%%%%%%%%%%%%%%%%%%%%%%
%%%%%%%%%%%%%%%%%%%%%%%%%%%%%%%%%%

\begin{frame}[t]\frametitle{Periodic points}


$$\a(x)=\varphi(fx)-\varphi(x) \qquad (\ast) $$
%\pause
{\color{blue}{\bf Corollary.}}\, 
$f$ preserves a volume form
if and only if 

\hskip1.9cm$ \det Df^n(p)= 1\,$  whenever $\,f^n p=p$.

\pause

\vskip.2cm
Fix a volume form $\omega$ and 
let $J_\omega(x)$ be the Jacobian of $f$ w.r.t. $\omega$. 
\vskip.2cm

If $\omega '=\frac{1}{c(x)}\omega$ then $\,J_{\omega '}(x)=c(fx)^{-1}c(x)J_\omega(x)$.
\vskip.2cm

\pause

$f$ preserves $\omega '$ $\Leftrightarrow$ $\,J_{\omega '}(x)=1$ for all $x$ 
$\Leftrightarrow$ $J_\omega(x)=c(fx)\, c(x)^{-1}$
\vskip.2cm



$\Leftrightarrow$ $(\ast) $ with 
 $\a (x) =\log J_\omega$ and $\varphi(x)=\log c(x)$. 
\vskip.2cm

  
 \pause
 A solution exists if and only if 
$$0=\sum_{i=0}^{n-1} \a(f^ip) =\log \det Df^n(p) \quad \text{whenever}  p=f^np.$$



\vskip.1cm 

\end{frame}

%%%%%%%%%%%%%%%%%%%%%%%%%%%%%%%%%%


\begin{frame}[t]\frametitle{Linear cocycles over hyperbolic systems}
$f:\M\to \M\;$  a transitive  Anosov diffeomorphism;
\vskip.3cm

%\pause

$P : \,\E \to \M\;\,$ a finite dimensional vector bundle 

\hskip2.3cm with H\"older continuous Riemannian metric;
\vskip.3cm

\pause
$F:\,\E \to \E\;\;\;$ a H\"older continuous linear cocycle over $f$, i.e.
\vskip.3cm
\begin{tabular}{lp{0cm}l}
$
\begin{array}{ccc}
\E & \overset{F}{\rightarrow} &\E  \\
 {\text{{\it \footnotesize P}}}\downarrow  &  &  {\text{{\it \footnotesize P}}}\downarrow  \\
\M & \overset{f}{\rightarrow}  & \M 
 \end{array} $&&
$\begin{array}{c}\text{and $\,F_x : \E_x \to \E_{fx}\,$ is a linear isomorphism \hskip3.6cm} \\\text{ which depends H\"older continuously on $x$.\hskip4cm}
 \end{array}$
\end{tabular}
\vskip.4cm

%\pause    

\end{frame}

%%%%%%%%%%%%%%%%%%%%%%%%%%%%%%%%%%

%%%%%%%%%%%%%%%%%%%%%%%%%%%%%%%%%%


\begin{frame}[t]\frametitle{Linear cocycles over hyperbolic systems}

$F:\,\E \to \E\;\;\;$ a H\"older continuous linear cocycle over $f$.
\vskip.3cm


{\color{blue}{\bf Example:}} \, {\bf Derivative cocycle.} 
\vskip.05cm
The differential $Df$ is a cocycle on the tangent bundle $\E=T\M$. \\
If $\E'$ is a $Df$-invariant sub-bundle, then $Df|_{\E'}$ is also a cocycle.

\vskip.6cm
\pause

{\color{blue}{\bf Example:}} {\bf Trivial bundle.} 

 $\E = \M\times \R^d$, 
\hskip.3cm so $\E_x = \E_{fx}=\R^d$, $\quad F_x\in GL(d,\R)$,
\vskip.05cm
 and $F$ can be viewed as a H\"older function $F: \M \to GL(d,\R)$.
\vskip.6cm
\pause
More generally, let $f:\M\to \M\;$ be a hyperbolic system:
\vskip.1cm
Anosov diffeomorphism, locally maximal hyperbolic set, \\
or a symbolic system such as subshift of finite type.
\vskip.2cm

%\pause    
\vskip.4cm
\end{frame}

%%%%%%%%%%%%%%%%%%%%%%%%%%%%%%%%%%



\begin{frame}[t]\frametitle{Periodic data of a cocycle}
$F:\,\E \to \E\;\;\;$ be a H\"older continuous linear cocycle over $f$.

For a periodic point $\,p=f^np\,$ in $\M$, consider the return map
\vskip.2cm
$
\hskip2cm F^n_p = F_{f^{n-1}p}\circ \cdots \circ F_{fp} \circ F_p : \;\; \E_p \to \E_p
$
\vskip.4cm
%\pause    
{\color{blue}{\bf Question:}}\,  What can be said about $F$ based on its

\hskip2cm  {\bf periodic data}  $ \,\{F_p^n\}\;$? 
\vskip.2cm
\pause    
In particular, what can be said about $F$ based on Lyapunov exponents
at periodic points?
\vskip.2cm
%\pause

The Lyapunov exponents of $F$ at a periodic point $p=f^np$ \\
are given by eigenvalues of $F^p_n$: 
$$\lambda ^{(p)}_i = \frac 1n \, \log \, | i^{th} \text{ eigenvalue of } F^n_p |$$



%\vskip.2cm
%$\bullet$ $\;\lambda_+(F,p)-\lambda_ -(F,p) \le \gamma$.


\end{frame}
%%%%%%%%%%%%%%%%%%%%%%%%%%%%%%%%%%
%%%%%%%%%%%%%%%%%%%%%%%%%%%%%%%%%%

\begin{frame}[t]\frametitle{ Oseledets' Multiplicative Ergodic Theorem (1965)}

 
Let $f$ be an ergodic measure
preserving transformation \\ of a Lebesgue probability measure space
 $(X,\mu)$ and let \\$F: X \to GL(d,\R)\; $ be a measurable cocycle over $f$.
 \vskip.2cm   %\pause
 If $\;\log\|F_x\|,\;\log\|F_x^{-1}\| \in L^1(X,\mu)\;$ then   there exist
 \vskip.05cm
numbers $\lambda_1 < \dots < \lambda_l$, an $f$-invariant set 
$\mathcal{R}$ of full measure, \\ and an $F$-invariant decomposition of $\R^d$ 
for  $x\in \mathcal{R}$ 
\vskip.2cm
$
\hskip2.5cm \R^d_x= E_{\lambda_1}(x)\oplus\dots\oplus E_{\lambda_l}(x)
$
\vskip.2cm
 such that for any nonzero
$v\in E_{\lambda_i}(x)$, 
$ 
\; \underset{n\to{\pm \infty}}{\lim} \frac 1n \log\| F^n_x v \|=  \lambda_i .  
$
\vskip.4cm
The numbers $\,\lambda_1,\dots,\lambda_l\,$ are called 
the {\bf Lyapunov exponents} of $F$. 


\end{frame}


%%%%%%%%%%%%%%%%%%%%%%%%%%%%%%%%%%



\begin{frame}[t]\frametitle{Periodic approximation of Lyapunov exponents}



\begin{theorem} [B.K. 2008]

Let $\;f:\M \to \M$ be a hyperbolic system. \\
Let $\;F:\,\E \to \E\;$ be a H\"older continuous linear cocycle over $f$. \\
Let $\,\mu\,$ be  an ergodic invariant measure for $f$. 
\\
%\pause
\ \\
Then the Lyapunov exponents $\lambda_1 \le ... \le \lambda_d$ 
 of $F$ with respect 
 to $\mu$ (listed with multiplicities) can be approximated 
by the Lyapunov exponents of $F$ at periodic points. 
 \\
%\pause
\ \\
More precisely,
for any $\e >0$ there exists a periodic point $p \in \M$ for which the Lyapunov 
exponents $\lambda_1^{(p)} \le ... \le \lambda_d^{(p)}$ of $F$ satisfy
$|\lambda_i-\lambda_i^{(p)}|<\e\,$ for $i=1, \dots , d$.


\end{theorem} 

\end{frame}


%%%%%%%%%%%%%%%%%%%%%%%%%%%%%%%%%%
%%%%%%%%%%%%%%%%%%%%%%%%%%%%%%%%%
\begin{frame}[t]\frametitle{Uniform growth estimates for cocycles}



\begin{corollary} 
Let $F$ be a H\"older linear cocycle over a hyperbolic system $f$. 
\vskip.2cm
Suppose that for each periodic point $p=f^np$ \\
the largest Lyapunov exponent of $F$ at $p$ is at most $\lambda$. 
\\
%\pause
\ \\
Then for every $\e > 0$ there exists a constant $C_\e$ 
such that for all $x \in M$ and $n \in \N$
$$
 \|  F^n_x \| \le C_{\e} e^{(\lambda  +\e)n}  
$$
\end{corollary}

The largest  Lyapunov exponents of $F$ with respect to $\mu$:
$$
\lambda_+(F,\mu)=  \lim_{n \to \infty} \frac 1n \log \| F_x ^n \| 
\quad \quad\quad\;\;\;\text{for } \mu \text { a.e.} \; x\in \M .\hskip.4cm
$$


\end{frame}

%%%%%%%%%%%%%%%%%%%%%%%%%%%%%%%%%%


%%%%%%%%%%%%%%%%%%%%%%%%%%%%%%%%%%

\begin{frame}[t]\frametitle{ Conformality}

{\bf Quasiconformal distortion}
\vskip.2cm
$\hskip.4cm K_F(x,n) = \|F^n_x\| \cdot \| (F^n_x)^{-1}  \| 
=\,${\Large $\frac{\max\,\{\,\|\,F_x^n\, (v)\,\| :\; v\in \E_x, \;\|v\|=1\,\}}
            {\,\min\,\{\,\|\,F_x^n\, (v)\,\| :\; v\in \E_x, \;\|v\|=1\,\}}$}

 \vskip.2cm          
%\pause    
$F$ is {\bf conformal}  on $\E\,$ if $\,K_F(x,n)=1\;$ 
for all $x\in \M$ and $n\in \Z$.
\vskip.3cm
\pause    
Three equivalent necessary conditions: whenever $f^np=p$,
\vskip.1cm
(1) $F_p^n$ 
is conformal with respect to an inner product on $\E_p$;
\vskip.1cm
(2) $F^n_p$ is diagonalizable over $\C$
with  eigenvalues equal in modulus;
\vskip.1cm
(3) $K_F(p,n) \le C(p)\;$ for all $n$ such that $f^np=p$.

\vskip.2cm

\pause    
\begin{theorem} [B.K., V. Sadovskaya]  
Suppose that the fibers of $\E$ are two-dimensional.
If $F$ satisfies \\ (1) or (2) or (3) at each periodic point,
then $F$ is \,{\bf conformal} \\ with respect to a H\"older continuous 
Riemannian metric on $\E$. 
\end{theorem}

\end{frame}


%%%%%%%%%%%%%%%%%%%%%%%%%%%%%%%%%%

\begin{frame}[t]\frametitle{ Conformality in higher dimension}

The Theorem does not hold in dimension $\ge 3$:
\vskip.2cm

There exists $F$ such that at every periodic point

$F^n_p$ is {\em isometric}\, with respect to an inner product on $\E_p$,

but $F$ is not conformal with respect to any continuous metric.

\vskip.4cm
\pause    

\begin{theorem} [B.K., V. Sadovskaya]  \label{periodic} 
If there exists a constant $C_{per}$ such that 
\vskip.2cm
$ \hskip1.8cm
K_F(p,n)  \le C_{per} \quad\text {whenever }f^np=p,
$ 
\vskip.2cm  
then $F$ is \,{\bf conformal}  with respect to a H\"older continuous 
Riemannian metric on $\E$. \\
\end{theorem}


\end{frame}
%%%%%%%%%%%%%%%%%%%%%%%%%%%%%%%%%%

%%%%%%%%%%%%%%%%%%%%%%%%%%%%%%%%%%

\begin{frame}[t]\frametitle{ Isometries}

If $F$ is an {\bf isometry}, then whenever $f^np=p$,
\vskip.2cm
(1) $F_p^n$ 
is an isometry with respect to an inner product on $\E_p$;
\vskip.2cm
(2) $F^n_p$ is diagonalizable over $\C$, its eigenvalues 
are of modulus 1;
\vskip.2cm
(3) $ \max\,\{\|F^n_p\|, \|(F^n_p)^{-1}\|\} \le C'(p)\;$   
for all $n$ such that $f^np=p.$
\vskip.2cm
\pause    
\begin{theorem} [B.K., V. Sadovskaya] 
For two-dimensional fibers:
If $F$ satisfies  (1) or (2) or (3) \\at each periodic point, 
then $F$ is an\, {\bf isometry}
 with respect \\ to a H\"older continuous 
Riemannian metric on $\E$. 
\vskip.3cm
\pause

In general: if there exists a constant $C'_{per}$ such that 
\vskip.2cm
$ \hskip1cm\max\,\{\|F^n_p\|, \|(F^n_p)^{-1}\|\} \le C'_{per}  
  \quad \text{whenever }f^np=p, $
\vskip.2cm  
then $F$ is an\, {\bf isometry}. 


\end{theorem}

\end{frame}
%%%%%%%%%%%%%%%%%%%%%%%%%%%%%%%%%%
\begin{frame}[t]\frametitle
{Isometric  at periodic points, but not conformal, in dim 3}

\begin{example}[dim$\,\E_x \ge 3$]
There exists $F:\E\to\E$ such that\, 
whenever $f^np=p$

$F^n_p$ is {\bf isometric}\, with respect to an inner product on $\E_p$, but 

$F$ is {\bf not conformal}\, with respect to any continuous metric on $\E$.
\end{example}
\pause    
\vskip.2cm
 
Let $\E=\M\times \R^3, \qquad   F_x=${\small $\left[ \begin{array}{ccc}
             \cos \a(x)  & -\sin \a(x)       & \; \e       \\
             \sin \a(x)   & \;\;\;\cos \a(x) & \; 0         \\
             0               & 0                    & \; 1    \\             
           \end{array} \right] .$}
\vskip.2cm
\pause    
Let  $S$  be  a closed $f$-invariant set in $\M$ 
without  periodic points;
\vskip.15cm

 $\a : \M\to \R$, 
 $\;\;\;\a(x)=0\,$ for $x\in S\;$ and $\;0< \a(x) \le \e$ for $x\notin S\;$
\vskip.3cm
%\pause    
% for every periodic $p\,$ and its minimal period $n$,
%\vskip.1cm
%$\a(p,n)=\a(f^{n-1}p)+ \dots +\a(p)\ne \pi k$.
 %\vskip1cm
   

\begin{tabular}{lp{0cm}l}
\noindent For $x \in S$,   {\small $\;\;F^n_x=\left[ \begin{array}{ccc}
             1  & 0       & {\mathbf n}\e       \\
             0   & 1 & 0        \\
             0               & 0                    & \; 1    \\             
           \end{array} \right] $} &&
$\begin{array}{c} 
K_F(x,n) \to \infty \text{ as } n\to\infty \Longrightarrow\, \hskip1.1cm \\
 F \text{ is not conformal 
w.r. to any } \hskip1cm \\ \text {continuous
Riemannian metric.}\hskip1cm
\end{array}$
\end{tabular}
\vskip.3cm

\pause
At $p=f^n p$, $\;\;F^n_p\,$ is diagonalizable
with eigenvalues of modulus 1.



\end{frame}

%%%%%%%%%%%%%%%%%%%%%%%%%%%%



%%%%%%%%%%%%%%%%%%%%%%%%%%%%%%%%%


\begin{frame}[t]\frametitle
{ Local rigidity for Anosov diffeomorphisms}

$f$ is an Anosov  diffeomorphism and $\;g$ is its $C^1$-small perturbation
\vskip.05cm

then $\;f=h^{-1}\circ g\circ h,\;\;$ and $h$ is only H\"older continuous in general.
\vskip.3cm
\pause    
If $h$ is smooth, then for any periodic point $p=f^np$,
\vskip.1cm
$\hskip2cm  Df^n_p = (Dh_p)^{-1} \circ D g^n_{h(p)} \circ Dh_p.$
\vskip.25cm
\pause
$f$ is called {\bf locally rigid}\, if conjugacy  $\,Df^n_p\,$ and $\,Dg^n_{h(p)}\,$ implies smoothness of $h\,$
for every $C^1$-small perturbation $g$. 
\vskip.3cm
\end{frame}

%%%%%%%%%%%%%%%%%%%%%%%%%%%%%%%%%

\begin{frame}[t]\frametitle
{ Local rigidity for Anosov automorphisms}

  
{\color{blue}{\bf Question:}}\, Is an Anosov diffeomorphism $f$ locally rigid?
\vskip.3cm
%\pause    
{\bf Yes}\, if \,dim$\,E^u\,$=\,dim$\,E^s =1$ (de la Llave),\,\, 
\vskip.3cm
{\bf No}\,\,\, in higher dimension\, (de la Llave).
\pause
\vskip.3cm
{\bf Yes}\, if $Df^n|_{E^s(p)}$,  $Df^n|_{E^u(p)}$ are $k(p)\,Id\;$
\vskip.1cm
\hskip.8cm (de la Llave; \, B.K, V. Sadovskaya)
\pause
\vskip.3cm
{\bf Yes}\, if $Df^n|_{E^s(p)}$ and $Df^n|_{E^u(p)}$ are conformal\, and 
\vskip.1cm
\hskip.8cm  $\dim E^u = \dim E^s=2\hskip.7cm$  
(B.K, V. Sadovskaya)

\end{frame}

%%%%%%%%%%%%%%%%%%%%%%%%%%%%%%%%%

\begin{frame}[t]\frametitle
{ Local rigidity for Anosov automorphisms}


\begin{theorem} [Gogolev, B.K., Sadovskaya]
Let $\,L:\T^d\to\T^d$ be an irreducible Anosov automorphism \\
such that no three of its eigenvalues have the same modulus.\\
Then $L$ is \,{\bf locally rigid}, %\pause 
more precisely
\vskip.2cm


If $g$ is a $C^1$-small perturbation of $L$ 
with conjugate periodic data,\\
then $g$ is $C^{1+\text{H\"older}}$ conjugate to $L$.

\end{theorem}


\pause

\begin{proposition} Toral automorphisms satisfying the
assumptions are\, {\bf generic}: \\
the proportion of matrices $L$ in $SL(d,\Z)$ with  $\| L\| \le T$\\
that do {\bf not} satisfy the assumptions can be estimated by \\
$cT^{-\delta}\,$ for some $\,\delta >0$.
\end{proposition}


\end{frame}

%%%%%%%%%%%%%%%%%%%%%%%%%%%%%%%%%%



\begin{frame}[t]\frametitle{ Role of conformality}


Let $L$ be an Anosov automorphism of $\T^d$. $\;L \in SL(d,\Z)$.
\vskip.1cm

Let $1<\rho_1 <\rho_2< \dots <\rho_m\;\;$  be the distinct moduli of \\

\hskip4.9cm  the unstable eigenvalues of $L$.
\vskip.2cm

The corresponding splitting:  $\;E^{u,L}=
E_1^L \oplus E_2^L \oplus \dots \oplus E_m^L$ 
\pause    
\vskip.2cm
For the perturbation $g$: $\hskip.85cmE^{u,g} =
E_1^g \oplus E_2^g \oplus \dots \oplus E_m^g$
\vskip.3cm

\pause


Since $L$ is irreducible, it is diagonalizable over $\C$, 

and hence $L$ is conformal on each $E_i^L$.
\vskip.2cm

\pause
As the periodic data  are conjugate, the cocycle 
$Dg|_{E_i^g}$ is \\ conformal at the periodic points.
Since $\dim E_i^g= \dim E_i^L\le 2$,\\ the Theorem implies
 that $g$  is conformal on $E_i^g$. 

\vskip.2cm
Conformality of $g$ allows to establish smoothness of $h$ along $E_i^L$.

\end{frame}

%%%%%%%%%%%%%%%%%%%%%%%%%%%%%%%%%%

%%%%%%%%%%%%%%%%%%%%%%%%%%%%%%

\begin{frame}[t]\frametitle
{$GL(2,\R)$-valued cocycles with one exponent}
 
 \vskip.2cm
% Skip the next:
%Then $F$ preserves either a H\"older continuous one-dimensional 
%sub-bundle or a H\"older continuous conformal structure on $\E$.

\begin{proposition}[V. Sadovskaya] 
Let $F:\M\to GL(2,\R)\,$ be an orientation-preserving cocycle such that
for each  $p=f^np$, 
the eigenvalues of 
$F^n_p $
are equal in modulus.
%\pause    
\vskip.2cm
 Then, possibly after passing to a double cover,
$F$ is conjugate to
\vskip.3cm

$ \;\;k(x)\left[ \begin{array}{cc}1 & \b(x) \\  
 0 & 1 \end{array} \right] \text{ or }\;
 k(x)\Id\quad \text{or}\quad k(x) \left[ \begin{array}{cc} \cos \b(x) & -\sin \b(x) \\ 
        \sin \b(x)& \;\;\;\cos \b(x) \end{array} \right]
$
\vskip.4cm
\end{proposition}

% add counterexamples and/or comments

\end{frame}
 %%%%%%%%%%%%%%%%%%%%%%%%%%
 


\begin{frame}[t]\frametitle{ Structural Theorem}

$F:\E\to \E$ a H\"older continuous linear cocycle, $\;\dim \E_x=d$.

%\pause

\begin{theorem} [B.K, V. Sadovskaya]
Suppose that for each periodic point $p = f^np$, the eigenvalues \\
of $F^n_p$ are equal in modulus. \pause
Then there exists a flag of H\"older continuous $F$-invariant sub-bundles 
\vskip.15cm
$\hskip2cm \E^1 \subset ...  \subset \E^{k-1} \subset \E^k =\E \quad\;$
\vskip.15cm

and H\"older continuous Riemannian metrics on $\E^1$ and \\
on the factor bundles $\E^{i+1}/\E^{i}$, $\;i=1, ... , k-1$, such that
\begin{itemize}
\item $F|_{\E_1}$ is conformal and  
\item the factor-maps induced by $F$
on  $\E^{i+1}/\E^{i}$ are conformal.
\end{itemize}
 \end{theorem} 

\vskip.2cm
%\pause

If the flag is trivial then $F$ is conformal on $\E$.


\end{frame}


\begin{frame}{Structural Theorem -- Special Case}
If there are $d$ continuous vector fields which give bases 
for all $\E^i$, 
%\vskip.1cm
then $F$ is H\"older 
cohomologous to a cocycle of the form
\vskip.5cm

\begin{tabular}{lp{0cm}l}
{\small $
 \quad \left[ \begin{array}{cccc}
 A_1(x) & \ast & \ldots & \ast \\
 0  & A_2(x) & \ddots &\vdots \\
 \vdots & \ddots & \ddots & \ast \\
 0 & \ldots & 0 & A_k(x)
 \end{array} \right]$}
 & &
$\begin{array}{c} \text{
$A_i(x)=l_i(x)O_i(x)$} \hskip.75cm \\  \text{is a scalar 
 multiple of } \\ \text{an orthogonal matrix.}
 \end{array}$
\end{tabular}
\vskip4cm
\end{frame}



%%%%%%%%%%%%%%%%%%%


%%%%%%%%%%%%%%%%%%%%%%%%%%%%%%%%%%

\begin{frame} [t]\frametitle{Polynomial growth}

{\bf Quasiconformal distortion}\, of $F$
\vskip.2cm
$
K_F(x,n)= \|F^n_x\| \cdot \| (F^n_x)^{-1} \| =\;
${\Large $\frac{\max\,\{\,\|\,F_x^n\, (v)\,\| \,:\; \,v\in \E_x, \;\|v\|=1\,\}}
            {\,\min\,\{\,\|\,F_x^n\, (v)\,\| \,:\; \,v\in \E_x, \;\|v\|=1\,\}}
$}
%\pause
%\vskip.2cm
\begin{theorem}
Suppose that for each periodic point $p = f^np$, \\ the eigenvalues 
of $F^n_p$ are equal in modulus. 
\vskip.2cm

Then  $\;\mathbf {K_F(x,n) \le Cn^{2m}}\,$ for all $x\in \M$ and $n\in \N$. 
\vskip.3cm
\pause
If the eigenvalues
of $F^n_p$ are of modulus 1 whenever $p = f^np$,
\vskip.2cm 
then $\,\mathbf{\| F^n_x\| \le Cn^{m}}\,$ for all $x\in \M$ and $n\in \N$. 
 \end{theorem} 
\vskip.2cm
$\mathbf m=\,$ the number of non-trivial sub-bundles in the flag
$\;\le d-1$.
\end{frame}

%%%%%%%%%%%%%%%%%%%%%%%%%%%%%%%%%%
%%%%%%%%%%%%%%%%%%%%%%%%%%%%%%%%%%
\end{document}
%%%%%%%%%%%%%%%%%%%%%%%%%%%%%%%%%%
%%%%%%%%%%%%%%%%%%%%%%%%%%%%%%%%%%


%%%%%%%%%%%%%%%%%%%%%%%%%%%%%%%%%%