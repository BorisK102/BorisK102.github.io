\documentclass[12pt]{letter}
\textheight23cm
\textwidth17.65cm
\hoffset=-1.9cm
\voffset=-2.5cm
\addtolength{\textheight}{1.6cm}
\pagestyle{empty}



\begin{document}
\begin{center}
{\bf  {\large  MATH  403H $\;$ Honors Classical Analysis I  $\;$ Fall 2024}} 

\end{center}

\vskip.1cm
\begin{tabbing}
{\bf Class meets: $\hskip.5cm $} \= Tuesday, Thursday$\,$   \kill
{\bf Class meets:} \> Tuesday, Thursday$\,$ 1:35 - 2:50 p.m.   in 013 Huck Life Sciences Bldg  \\\\
{\bf Instructor:} \>  {\bf Boris Kalinin} \\
{\it \hskip.5cm E-mail:} \>   kalinin@psu.edu \\
{\it \hskip.5cm Office:} \>  338 McAllister  \\
{\it \hskip.5cm Phone:} \>  (814) 865-5181 \\
{\it \hskip.5cm  Office hours:} \>   Tuesday 10:25 - 11:50 or by appointment.
\end{tabbing}

{\bf Goals and Objectives:}
%http://bulletins.psu.edu/undergrad/courses/m/math/403h
We will develop foundations of  analysis in the framework of metric 
spaces and explore its applications with emphasis on function spaces.
Topics covered will include metrics and norms; open, closed, and compact 
sets in metric spaces; maps between metric spaces and their continuity properties; 
 spaces of functions; applications.

{\bf Prerequisite:} MATH 311M and MATH 312H (or 311W and 312, with permission).


{\bf Text:} {\it Real Analysis}\, by N. L. Carothers. 
ISBN-13: 978-0521497565 \\
We will cover most of the material in Chapters 3-8 and parts of Chapters 9-12.

% {\bf Course web page:}   {\tt http://www.personal.psu.edu/bvk102/403H/403H.html} \,\\ It will contain  the syllabus, daily schedule, homework assignments, exam study guides, etc.


 {\bf Canvas:}  The course materials, including the syllabus, daily schedule, homework assignments,  announcements, and exam study guides will be posted on {\it Canvas}.

%  {\bf Attendance:} You are expected to attend every class.   If you miss a class, you are responsible for learning the material covered and knowing the announcements made in class. 

{\bf Homework:} 
Homework assignments will be given on a weekly basis and (unless specified otherwise) will be due on Thursdays by the beginning of the class. No late homework will be accepted, except in cases of illness or emergency. Solutions of homework problems must be {\em neatly and clearly written in complete sentences and logically structured}. You may use without justification statements proved in class, results of previous homework assignments, and statements from the sections of the book that we already covered. You may discuss a general approach to a solution with other students, but you must write solutions on your own. 
{\it You may not copy solutions from any source and you may not use any online sites (e.g., Course Hero or Chegg), technologies (e.g., ChatGPT), tools, or sources.}
 Additionally, I will assign practice problems that will not be collected. I strongly encourage you to do all the suggested problems as this will help with understanding the course material and with exam preparation.


 
% {\bf Quizzes:}  There will be quizzes on Mondays at the beginning of the class. 
 % on the material covered during the previous week.  You may be asked to give definitions,  state theorems, answer true/false questions, and give examples.   There will be no make-up quizzes, but the lowest two scores will be dropped.

{\bf Exams:} There will be  two  midterm exams and a  two-hour comprehensive final exam. The dates of the midterm exams will be announced approximately two weeks in advance, and the final exam will be scheduled by the university.  All students must plan to take exams at the scheduled times. If you are unable to take a midterm you must notify me promptly: if there is a compelling reason for absence, your score will be replaced by the final exam score; otherwise, your score will be zero.




%\\ Class attendance and participation will be considered in determining borderline grades.



{\bf Grading Policy:} The final score and letter grade will be 
calculated as follows:

\hskip1cm
\begin{tabular}{ll}   
Homework: &  45\% \\ 
Exam 1: & 15\% \\             
Exam 2: & 15\% \\
Final Exam: & 25\% \\
\end{tabular} \hskip1cm
\begin{tabular}{ll}
A & at least 90\% \\
A-- & at least 87\% \\
B{+} & at least 84\% \\
B & at least 80\%  \\
\end{tabular}
 \hskip1cm
\begin{tabular}{ll}
B-- & at least 77\% \\
C+ & at least 74\%  \\
C & at least 70\%  \\
D & at least 60\%  \\ 
\end{tabular}
%\vskip.2cm 

\newpage



 {\bf Attendance:} You are expected to attend every class. 
  If you have to miss a class, please visit Canvas to find out what was covered and to check 
  for announcements. You will need to study the book or class notes to learn the material covered that day.\\
  {\it Please notify me if you have to miss more than one class.} \\
  Class attendance and participation will be considered in determining borderline grades.
 
\vskip.5cm

{\bf Academic Integrity}  is the pursuit of scholarly activity in an open, honest and responsible manner. It is a basic guiding principle for all academic activity at The Pennsylvania State University, and all members of the University community are expected to act in accordance with this principle. Consistent with this expectation, the University�s Code of Conduct states that all students should act with personal integrity, respect other students� dignity, rights and property, and help create and maintain an environment in which all can succeed through the fruits of their efforts.
Academic integrity includes a commitment by all members of the University community not to engage in or tolerate acts of falsification, misrepresentation or deception. Such acts of dishonesty violate the fundamental ethical principles of the University community and compromise the worth of work completed by others.

{\bf Disability accommodation:} 
Penn State welcomes students with disabilities into the University's educational programs. Student Disability Resources (SDR) website provides contact information for every Penn State campus {\tt http://equity.psu.edu/sdr/disability-coordinator}. For further information, please visit {\tt http://equity.psu.edu/sdr/}.
In order to receive consideration for reasonable accommodations, you must contact the appropriate disability services office, participate in an intake interview, and provide documentation 
({\tt http://equity.psu.edu/sdr/guidelines}). If the documentation supports your request for reasonable accommodations, your campus disability services office will provide you with an accommodation letter. Share this letter with your instructors and discuss the accommodations with them as early as possible. You must follow this process for every semester that you request accommodations.

{\bf Counseling and psychological services:} 
Many students  face personal challenges or have psychological needs that may interfere with their academic progress, social development, or emotional wellbeing. The university offers a variety of confidential services to help you through difficult times, including individual and group counseling, crisis intervention, consultations, online chats, and mental health screenings. These services are provided by staff who welcome all students and embrace a philosophy respectful of clients� cultural and religious backgrounds, and sensitive to differences in race, ability, gender identity and sexual orientation. \\
$\hphantom{} \hskip.5cm$ {\it  Counseling and Psychological Services at University Park  (CAPS)} \\
$\hphantom{} \hskip1cm$ {\tt http://studentaffairs.psu.edu/counseling/} \, 814-863-0395 \\
$\hphantom{} \hskip.5cm$ {\it Penn State Crisis Line}\, (24 hours/7 days/week):\, 877-229-6400\\
$\hphantom{} \hskip.5cm$ {\it Crisis Text Line}\,  (24 hours/7 days/week):\,  Text LIONS to 741741


{\bf Reporting Educational Equity Concerns:}
Penn State takes great pride to foster a diverse and inclusive environment for students, faculty, and staff. Acts of intolerance, discrimination, or harassment due to age, ancestry, color, disability, gender, gender identity, national origin, race, religious belief, sexual orientation, or veteran status are not tolerated and can be reported through Educational Equity via the University's Report Bias webpage \,
{\tt http://equity.psu.edu/reportbias/}


\end{document}


{\bf Homework}\,  will be assigned on Fridays
 and  collected {\it by 5 p.m. on Thursdays via Canvas.} \\
{\it No late homework will be accepted.}\, The lowest two homework scores will be dropped.\\
Solutions of homework problems must be neatly and clearly written or typed in complete sentences.
The solutions must be  logically structured, and all steps must be justified. You may use  
 statements proved in class and  results of previous homework  assignments.
You may discuss general approaches to a problem with me 
 and with other students, but you must write  solutions on your own.\\
 {\it You may not copy solutions from any source and you may not use any online sites (e.g., Course Hero or Chegg), technologies (e.g., ChatGPT), tools, or sources.}
  Additionally, I will assign practice problems that will not  be collected. 
I strongly recommend doing all suggested exercises as this will help
with understanding the course material and with exam preparation.


{\bf Exams:} There will be  two midterm exams and a two-hour cumulative final exam.\\
 The midterms will take place in class on {\it Monday, October 2,}\, and on {\it Monday, November 6.} \\
The final exam  will be scheduled by the university during the final exam week, {\it December 11\,-\,15.} \\
All students must plan to take exams at the scheduled times. If you are unable to 
attend a midterm exam you should notify me promptly. If there is a compelling reason 
for absence, such as illness or a family emergency, your score will be replaced by the 
final exam score. Otherwise, the score for the missed test will be zero.





