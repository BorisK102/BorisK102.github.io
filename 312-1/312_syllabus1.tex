\documentclass[12pt]{letter}
\textheight23.3cm
\textwidth16.7cm
\hoffset=-1.5cm
\voffset=-3.2cm
\addtolength{\textheight}{2cm}
\pagestyle{empty}

\begin{document}

\hskip0.7cm {\bf  {\large $\;\;$MATH  312.001 $\;\,$ Concepts of Real Analysis $\,\;$ Spring 2025}} 


\vskip.3cm
%{\bf Class meets:} \, Monday, Wednesday, Friday\, 11:15 a.m.  - 12:05 p.m.  in  109 Boucke Bldg
{\bf Class meets:} \, Monday, Wednesday, Friday\, 10:10 a.m.  - 11:00 a.m.  in  109 Osmond Lab

{\bf  Instructor:}  {\bf Boris Kalinin}\\
{\it  E-mail:}\,\,   kalinin@psu.edu \\     
{\it Office:}\,\,\,\,   338 McAllister Building  \\
{\it  Office hours:}\,\,  Monday 1:30\,-\,3:10 p.m.;  \,  Wednesday 1:20\,-2:10 p.m.;  \,  and by appointment


This course provides rigorous development of the key concepts and results of real analysis. The topics include real numbers, sequences, infinite series, continuity, differentiation, and integration for functions of one real variable, and  the Fundamental Theorem of Calculus.\\
The course is focused on theory rather than computations. The students will learn and write proofs.  Mathematical exposition will be emphasized.

{\it Prerequisites:}\, Math 141 (required)\,  and Math 311W (recommended). 

 {\bf Canvas:}  The course materials, including the syllabus, daily schedule, homework assignments,  %announcements, 
 and exam study guides will be posted on {\it Canvas}.

{\bf Text:} {\it Elementary Analysis: The Theory of Calculus}\, by Kenneth A. Ross, second edition.\\
The students can access the text and download chapters, or the entire book, free of charge at\\
{\small {\tt http://link.springer.com.ezaccess.libraries.psu.edu/book/10.1007/978-1-4614-6271-2}}


{\bf Homework}\,  will be assigned on Fridays and collected {\it by 11 p.m.\,on Wednesdays via Canvas.} 
{\it No late homework will be accepted.}\, The lowest two homework scores will be dropped.\\
 Solutions of homework problems must be neatly and clearly written or typed, in complete sentences.
The solutions must be  logically structured, and all steps must be justified. You may use  
 statements proved in class and  results of previous homework assignments.
You may discuss general approaches to problems with me 
 and with other students, but {\it you must write  solutions on your own.
You may not copy solutions from any source and you  may not use any online sites (e.g., Course Hero or Chegg), 
technologies (e.g., ChatGPT), tools, or sources.} \\
Additionally, I will assign practice problems that will not  be collected. 
I strongly encourage you to do all suggested exercises since this will help
with understanding the course material.%  and with exam preparation.


{\bf Exams:} There will be  two midterm exams and a two-hour cumulative final exam.\\
The midterms will take place in class on {\it Monday, February 17,}\, and on {\it Monday, March 31}. \\
The final exam  will be scheduled by the university during the final exam week, {\it May 5-9.} \\
You must plan to take exams at the scheduled times. If you are unable to 
attend a midterm exam you should notify me promptly. %, before the exam if possible.  
 If there is a compelling reason 
for absence, such as illness or a family emergency, your score will be replaced by the 
final exam score. Otherwise, the score for the missed test will be zero.

 {\bf Quizzes:}  There will be short quizzes on Mondays at the beginning of the class
  on the previous week's material. You will be asked to give definitions, 
  state theorems, and answer true/false questions. 
  There will be no make-up quizzes, but the lowest two scores will be dropped.
  
%  {\bf Team worksheets:} A number of times during the semester,  the students will work on problems and questions in teams. Most of these worksheets will be collected and graded. Each team will submit just one set of solutions, and all its members will get the same score.
 
  {\bf Attendance:} You are expected to attend every class. If you miss a class, you are responsible for learning the material covered and knowing the announcements made in class. \\
   {\it  Class attendance and participation will be considered in determining borderline grades.} 


\newpage

{\bf Grading Policy:} The final score and letter grade will be 
calculated as follows:

\hskip1cm
\begin{tabular}{ll}   
Homework: &  30\% \\ 
Quizzes: & 15\% \\             
Each Midterm: & 15\% \\
Final Exam: & 25\% \\
\end{tabular} \hskip1cm
\begin{tabular}{ll}
A & at least 90\% \\
A-- & at least 87\% \\
B{+} & at least 83\% \\
B & at least 80\%  \\
\end{tabular}
 \hskip1cm
\begin{tabular}{ll}
B-- & at least 77\% \\
C+ & at least 73\%  \\
C & at least 70\%  \\
D & at least 60\%  \\ 
\end{tabular} 

\vskip.1cm

{\small
{\bf Academic Integrity}  is the pursuit of scholarly activity in an open, honest and responsible manner. It is a basic guiding principle for all academic activity at The Pennsylvania State University, and all members of the University community are expected to act in accordance with this principle. Consistent with this expectation, the University�s Code of Conduct states that all students should act with personal integrity, respect other students� dignity, rights and property, and help create and maintain an environment in which all can succeed through the fruits of their efforts.
Academic integrity includes a commitment by all members of the University community not to engage in or tolerate acts of falsification, misrepresentation or deception. Such acts of dishonesty violate the fundamental ethical principles of the University community and compromise the worth of work completed by others.

{\bf Disability accommodation:} 
Penn State welcomes students with disabilities into the University's educational programs. Contact information for the Student Disability Resources (SDR) office at University Park is at 
{\tt https://equity.psu.edu/offices/student-disability-resources/contact}. For further information, visit \, {\tt https://equity.psu.edu/offices/student-disability-resources}.
In order to receive consideration for reasonable accommodations, you must contact the SDR office, participate in an introduction meeting, and provide documentation described at \\
{\tt https://equity.psu.edu/offices/student-disability-resources/documentation}.\\
 If the documentation supports your request for reasonable accommodations, the SDR office will provide you with an accommodation letter. Share this letter with your instructors and discuss the accommodations with them as early as possible. You must follow this process for every semester that you request accommodations.

{\bf Counseling and psychological services:} 
Many students  face personal challenges or have psychological needs that may interfere with their academic progress, social development, or emotional wellbeing. The university offers a variety of confidential services to help you through difficult times, including individual and group counseling, crisis intervention, consultations, online chats, and mental health screenings. These services are provided by staff who welcome all students and embrace a philosophy respectful of clients� cultural and religious backgrounds, and sensitive to differences in race, ability, gender identity and sexual orientation. \\
$\hphantom{} \hskip.5cm$ {\it  Counseling and Psychological Services at University Park  (CAPS)} \\
$\hphantom{} \hskip1cm$ {\tt http://studentaffairs.psu.edu/counseling/} \, 814-863-0395 \\
$\hphantom{} \hskip.5cm$ {\it Penn State Crisis Line}\, (24 hours/7 days/week):\, 877-229-6400\\
$\hphantom{} \hskip.5cm$ {\it Crisis Text Line}\,  (24 hours/7 days/week):\,  Text LIONS to 741741


{\bf Reporting Educational Equity Concerns:}
Penn State takes great pride to foster a diverse and inclusive environment for students, faculty, and staff. Acts of intolerance, discrimination, or harassment due to age, ancestry, color, disability, gender, gender identity, national origin, race, religious belief, sexual orientation, or veteran status are not tolerated and can be reported through Educational Equity via the University's Report Bias webpage \,
{\tt http://equity.psu.edu/reportbias/}
}
\end{document}






